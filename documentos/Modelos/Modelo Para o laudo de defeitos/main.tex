\documentclass[12pt,a4paper,final]{report}
\usepackage[utf8]{inputenc}
\usepackage{amsmath}
\usepackage{amsfonts}
\usepackage{amssymb}
\usepackage{graphicx}
\usepackage{cmap}
\usepackage{amsmath}
\usepackage[portuguese]{babel}
\usepackage[
backend=biber,
style=alphabetic,
sorting=ynt
]{biblatex}
\addbibresource{references.bib}
\author{Luan Bodner \\ 
        \and Felipe Veiga Ramos \\
        \and Paulo Batista da Costa\\ 
        \and Josimar Loch}
\title{Modelo para o laudo de defeitos}

\begin{document}
\maketitle
\setcounter{page}{1}

\setcounter{chapter}{1}


\section{Modelo para o laudo de defeitos}
O modelo a seguir foi baseado no artigo \cite{ModelingBugReportQuality}, mais especificamente na seção "A MODEL OF BUG REPORT QUALITY", neste artigo os autores propõe:
\subsection{Gravidade auto-relatada}
Quando a apresentação de um relatório utilizando o sistema de rastreamento de Bugs Bugzilla, o usuário é solicitado para avaliar a gravidade do erro determinando um dos níveis de severidade: crítica extrema, crítica, grande, normal, menor, trivial, ou melhoria.
Tratamos essa variável como um fator ordenada, onde crítica extrema é a mais grave e melhoria é a menos grave. Um
problema com gravidade auto-relatada é que os usuários talvez não sigam as diretrizes que descrevem cada categoria. Pode ser
tentador, por exemplo, para que o usuário relate o nível de um bug de forma errônea por não ter olhado com muita atenção. Para testar esta hipótese, nós também monitoramos as alterações posteriores à gravidade. A seção \ref{altlogp} descreve isso em mais detalhes.
\subsection{Medidas de Legabilidade}
Esse modelo incorpora várias medidas básicas de legibilidade.
Estes incluem a fórmula Coleman-Liau, a fórmula Kincaid,
o Índice de Legibilidade automatizado, e o índice de POLUIÇÃO ATMOSFÉRICA, todos os quais são baseados em características superficiais tais como o número médio de sílabas por palavra ou palavras por frase. Nessa hipótese é que os erros que são mais difíceis de compreender será mais difíceis de tratar e será
abordada mais tarde. Estas medidas são aplicadas ao inicial
relatório de bug descrição usando GNU Estilo versão 1.10-RC4.
\subsection{Reputação do Commiter}
Para cada relatório de bug, definimos a sua "reputação" apresentado do seguinte modo:

\[Reputação =\frac{C \cap R}{C + 1}\]

Onde \textit{C} é o conjunto de todos os relatórios apresentados pelos que apresentou antes do relatório de erro que está sob consideração e \textit{R} é o conjunto de todos os relatórios de bugs que foram resolvidos e ou
marcados como a copia de um relatório resolvido. Em outras palavras, essa pontuação mede a "taxa de sucesso" do commiter antes da apresentação do relatório sob consideração.
\label{altlogp}
\subsection{Alterações ao Longo do Tempo}

Esse modelo inclui vários recursos que são medidos após a apresentação do relatório inicial. Esses recursos incluem:
\begin{itemize}
\item mudanças de severidade do bug - Nessa hipótese é que os relatórios de erros com a severidade alta, podem levar mais tempo para serem corrigidos. Para testar esta hipótese, o modelo inclui o número total de escala da gravidade e de desanuviamento dentro de um conjunto de determinados períodos de tempo.
\item Contagem de Comentários - 
com o passar do tempo, registramos o número total de comentários que foram postados em resposta ao relatório inicial de bug. Note que, em geral, comentários podem ser enviados por qualquer pessoa. Isso significa que a contagem de comentário poderia servir como um proxy para algumas noções de popularidade.
\end{itemize} 
\subsection{Características Eludidas}
Além das características acima mencionadas, que considerando as
mudanças prioritárias do bug, da mesma forma como as mudanças de severidade. Mudanças prioritárias são anotações de contabilidade geralmente internas feito por um desenvolvedor, uma vez que o erro foi atribuído para posterior análise, mudanças prioritárias acabam sendo pouco frequentes, no entanto, o efeito destas características são  insignificantes.

\medskip
\printbibliography
\end{document}