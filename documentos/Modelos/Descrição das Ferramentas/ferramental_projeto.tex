\documentclass[12pt,a4paper,final]{report}
\usepackage[utf8]{inputenc}
\usepackage{amsmath}
\usepackage{amsfonts}
\usepackage{amssymb}
\usepackage{graphicx}
\usepackage[portuguese]{babel}
\title{Projeto: Estrutura Ferramental}
\begin{document}
	\maketitle
	\begin{quotation}
		\section*{Introdução}
		\quad Esta seção está destinada a descrição das ferramentas utilizadas. Nos projetos  da disciplina de engenharia de software 2, nossa equipe de gerentes adotou como ferramenta única para controle  de trabalho o \textit{github}. Esta é a base do processo de elaboração do \textbf{PBR}, bem como, para todo o desenvolvimento dos 4 projetos da disciplina. 
		
		\section*{Ferramenta - Verificação do PBR}
	\quad	A verificação do \textbf{PBR} é realizada através dos repositórios dos projetos, presentes na plataforma \textit{github}. Esta consiste num site que disponibiliza repositórios versionados pela ferramenta \textit{git}. Isto implica na facilidade de conferir alterações documentais por esta contém dentro de suas funcionalidade o \textit{diff}, este permite a visualização de alterações entre versões de documentos (anteriores e atuais) através da realização de \textit{commit}.
		Dessa forma, 4 projetos foram criados na plataforma e estes são relacionados cada um a um respectivo projeto. A responsabilidade de cada repositório é atribuída a ao respectivo gerente do projeto. Assim, cobrar atitudes de membros e elaboração das partes do \textbf{PBR}, bem como, todo o desenvolvimento é função do gerente.
		Assim, cada projeto é controlado pelo seu gerente no repositório \textit{on-line} do projeto no \textit{github}.
	\section*{Ferramenta - Verificar a organização do repositório}
	\quad Para verificação da organização do repositório, o próprio \textit{github} oferece suporte. Ele lista toda a estrutura do projeto, desde a documentações a desenvolvimentos de código. Assim, o gerente (responsável) verifica e controla como o repositório se mantém. Em outras palavras, o gerente possui toda a liberdade de controlar como pastas e arquivos se situarão da forma mais conveniente.
	\quad Entretanto algumas recomendações são necessárias e poderão ser verificadas na seção a seguir.
	\section*{Ferramenta - Como organizar o repositório}
	\quad Para a organização do repositório é dada liberdade aos gerentes. São eles que devem ditar como seus repositórios serão organizados. Entretanto, em todos eles, as métricas de organização presentes nesta seção serão seguidas. São elas:
	\begin{itemize}
		\item Separar arquivos de código de arquivos relacionados à documentação.
		\item Dentre os arquivos de documentação: Separar os arquivos de documentação do projeto dos arquivos gerados durante o processo de \textbf{PBR}.
		\item Para documentação: Armazenar somente arquivos ``\textit{.tex}", se possível. 
		\item Para \textit{source-code}: Evitar a presença de código inútil -- responsabilidade de verificação do gerente a respeito do trabalho do programador.
		\item Manter a estrutura semântica minimamente organizada: no caso da linguagem JAVA, os pacotes deverão ser coerentes com suas aplicabilidades oferecidas (evita-se aqui as classes e pacotes ``.utils").
	\end{itemize}
	\section*{Ferramenta - Verificação do andamento de atividades}
	\quad Toda verificação de andamento de atividade também será acompanhada através da utilização do github. Por ele ser completo, no quesito ferramenta ele possibilita que os gerentes vejam com o decorrer do tempo as diferenças em código e andamento das tarefas ao longo do tempo. As tarefas são correlacionadas a \textit{issues} e assim dão o controle necessário para o andamento do projeto.
	\section*{Ferramenta - Registro de Tempo}
	\quad Para verificação do andamento de atividades serão utilizadas as \textit{issues} do \textit{github}. Nelas o gerente terá o delta, ou seja, a medida exata do tempo entre a abertura de uma \textit{issue} e seu respectivo fechamento. Tendo isso, é possível realizar o controle de tempo das tarefas: pegam-se as \textit{issues} correspondente a uma tarefa e somam-se os deltas. Isto dará o controle temporal do projeto. 
	
	\quad Por sua vez, a utilização de \textit{milestones} nos permitirá ter controle sobre os estágios de desenvolvimento do projeto. Este é uma forma de controlar os marcos principais durante a fase de desenvolvimento da documentação e do projeto como um todo.
	\end{quotation}
	
	
\end{document}