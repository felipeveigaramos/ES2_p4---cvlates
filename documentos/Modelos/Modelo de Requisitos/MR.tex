\documentclass[12pt,a4paper,final]{report}
\usepackage[utf8]{inputenc}
\usepackage{amsmath}
\usepackage{amsfonts}
\usepackage{amssymb}
\usepackage{graphicx}
\usepackage[portuguese]{babel}
\author{Luan Bodner do Rosário}
\title{Modelo de Documento de Requisitos}

\begin{document}
\maketitle
\section*{Introdução}

Este documento tem o objetivo de definir um modelo claro e simples para auxiliar na criação de um documento de requisitos limpo e compreensível.
Para cumprir esta meta, definimos o formato em que os requisitos devem ser apresentados, o contexto em que o projeto deve ser definido e a estrura em que o documento deve ser organizado.

Esta seção deve mostrar uma breve explicação do projeto que deverá ser desenvolvido. Esta seção também deve explicar qual é necessidade o produto em questão e em qual contexto ele deverá ser utilizado.

\section*{Estrutura do Projeto}

\subsection*{Requisitos não-funcionais}

Descrição das ferramentas que poderão ser utilizadas caso determinadas funcionalidades sejam específicas dessas ferramentas. Além disso, o documento deve deixar clara a restrição de \textit{hardware}, ou seja, em que contexto de trabalho o produto será utilizado.
Esses requisitos devem ser enumerados e ter um identificador único, como mostrado no exemplo.
\begin{enumerate}
\item[RF1] Implementação feita na linguagem Ruby.
\item[RF2] Deve ser um aplicativo Web.
\end{enumerate}

\subsection*{Requisitos funcionais}

Parte principal do documento de requisitos.\\
Esta seção deve cobrir uma série de requerimentos para ser considerada satisfatória.\\
Os requerimentos estruturais do texto são:

\begin{itemize}
\item Os requisitos devem possuir um identificador único.

\item Os requisitos devem ser atômicos,
definidos em uma única sentença de forma sucinta.
Um exemplo de requisito funcional dado a seguir:
\begin{enumerate}
\item O sistema deve permitir o \textit{login} de funcionários.
\end{enumerate}

\item Os requitos devem ser consistentes entre sí e completos.

\item Cada funcionalidade deve ser possível dentro das restrições da equipe e ferramentas.

\item Os requisitos não mudam de acordo com a interpretação do leitor.

\end{itemize}

Detalhes quanto às entradas e especificações dos tipos de dados devem ser feitos em uma seção separada do documento e não junto com os requisitos.\\

\subsection*{Definição de dados}

Nesta seção, é necessário definir os atributos do projeto, os seus tipos, a definição de como as partes do sistema devem se comunicar.\\
Um exemplo disso seria a definição dos dados necessários para a criação de uma conta no sistema, ou os dados necessários(e o tipo de cada um desses dados) para acessar o sistema e suas interfaces.

\subsection*{Finalização do produto}

O documento deve incluir uma visão geral do produto finalizado, seja de forma textual ou visual, para aqueles que querem verificar as necessidades do projeto sem precisar ler e compreender todos os requisitos.\\

\section*{Conclusão}

Obviamente, a escala do projeto vai influenciar o nível de detalhes e o tamanho do requerimentos.\\
Independente do tamanho do projeto, é necessário seguir um modelo para que o produto desenvolvido esteja o mais próximo possível da visão do cliente e seja capaz de suprir as suas necessidades.\\
O documento de requisitos também deve ser capaz de gerar discussões entre o cliente e o gerente do time de desenvolvimento para que as funcionalidades do sistema sejam ajustadas e melhorias sejam feitas antes do processo de desenvolvimento começar.
Esta seção deve cobrir considerações finais sobre o projeto para poder levantar essas discussões.

\end{document}
