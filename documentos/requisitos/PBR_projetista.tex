\documentclass[12pt,a4paper]{report}
\usepackage[utf8]{inputenc}
\usepackage{amsmath}
\usepackage{amsfonts}
\usepackage{amssymb}
\usepackage{graphicx}
\usepackage[portuguese]{babel}
\author{Paulo Batista da Costa}
\title{PBR - Projeto CVlates : Visão do Projetista}
\begin{document}
\maketitle
\tableofcontents
\begin{quotation}
\newpage
\section{Introdução}
\section{Definições de elementos}
\textit{Todos os elementos (dados, estruturas de dados e funções) necessários foram definidos?}  \vspace{0.3cm}\\ 

 Sim. O documento especificou corretamente as funções necessárias para que seja garantida a corretude da implementação proposta. Através do que foi especificado é possível criar diagramas de caso de uso de forma clara. 
\section{Especificação de interfaces}
\textit{Todas as interfaces são especificadas e consistentes?} \vspace{0.3cm}\\ 

Não. O documento não especifica explicitamente toda a comunicação entre as partes do sistema, cabendo ao projetista maior esforço para dividi-lo  em módulos e identificar a comunicação entre eles. 
\section{Tipo de dados}
\textit{Foi possível definir todos os tipos de dados? (ex: unidades e precisão)}\vspace{0.3cm}\\ 

Sim, pois como os dados fazem parte do contexto do mundo real a identificação da natureza dos dados é implícita. Implícita, porém de fácil dedução por parte do projetista.
\newpage
\section{Disponibilidade de informações necessárias}
\textit{Todas informações necessárias para o projeto estão disponíveis?}\vspace{0.3cm}\\

Parcialmente. As informações do próprio projeto a respeito das funções e unidades estão presentes, há também a referência para outro sistema implementado (com sucesso) que serve de embasamento para o desenvolvimento da ideia de projeto. Entretanto, a respeito da integração com sistemas externos há falha em suas descrições. Perguntas como ``Quais sistemas externos?", ``Como ocorrerá a integração com demais plataformas" não puderam ser respondidas. 
\section{Especificação}
\textit{Está tudo especificado? (há alguma especificação funcional/requisito faltando?)}\vspace{0.3cm}\\

Não. Os requisitos foram todos escritos na área cabível.
\section{Análise crítica}
	\textit{Há algum ponto em que você não está certo sobre o que deve ser feito devido à falta de clareza na especificação do requisito/especificação funcional?}\vspace{0.3cm}\\
	Sim, Um exemplo disto é a não especificação de entradas muitas vezes descritas como ``palavra-chave" mas sem especificação do que de fato tal termo significa.
	\newpage
	\textit{A especificação funcional/requisito faz sentido considerando o que você sabe sobre o assunto ou considerando o que foi especificado nas descrições gerais/introdução?}\vspace{0.3cm}\\
	
	Faz sentido, porém seu conteúdo é redundante. Tudo o que foi escrito no documento poderia ser sintetizado numa documentação menor. Fora isso outro defeito foi lançar termos sem explicação razoável - como mencionado em questionamento anterior. Apesar disso, é um documento que transparece a ideia do contexto geral do produto a ser desenvolvido.
	
\end{quotation}
\end{document}