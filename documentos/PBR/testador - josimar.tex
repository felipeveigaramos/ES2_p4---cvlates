\documentclass[12pt,a4paper]{report}
\usepackage[utf8]{inputenc}
\usepackage{amsmath}
\usepackage{amsfonts}
\usepackage{amssymb}
\usepackage{graphicx}
\usepackage[portuguese]{babel}
\author{Josimar Loch}
\title{PBR - (P4) : Visão do Testador}
\begin{document}
\maketitle
\tableofcontents
\begin{quotation}
\newpage
\setcounter{section}{0}
\section{Introdução}
Sistema para centralizar dados de profissionais, estudantes, pesquisadores e docentes das áreas de Ciência e Tecnologia do país no estilo currículo lates.
\section{Funções de testador}
Validar se os requisitos estão ok e fazem sentido com a definição do projeto.
\section{Questões do Testador}
\subsection{Você tem toda informação necessária para identificar o item a ser testado? Você pode gerar um bom caso de teste para cada item?}
Sim, na grande maioria esta bem especificado, porem não será possível gerar caso de testes para todos.
\subsection{Você tem certeza de que os teste gerados fornecerão os valores corretos nas unidades corretas}
Sim, está razoavelmente claro cada tipo de teste e suas saídas.
\subsection{Existe uma outra interpretação dos requisitos de forma que o programador possa estar se baseando nela?}
Como as especificações estão relativamente claras é provável que não, porem caso não haja uma boa iteração possível que haja discrepâncias.
\subsection{Existe um outro requisito para o qual você poderia gerar um caso de teste similar, mas que poderia levar a um resultado contraditório?}
É provável que isso possa acontecer em relação ao requisito de atualização de dados, pois não fica muito claro quais dados devem ou podem ser atualizados.
\subsection{ A especificação funcional ou de requisitos faz sentido de acordo com aquilo que você conhece sobre a aplicação ou a partir daquilo que está descrito na especificação geral?}
Sim, especificações funcionais bem coesas com bom nível de detalhamento na sua grande maioria.
\end{quotation}

 \textit{Controle de esforço: 06/04 das 1:30 ás 2:45.}
\end{document}