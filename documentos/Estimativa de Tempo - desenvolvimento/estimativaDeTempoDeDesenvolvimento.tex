        \documentclass[12pt,a4paper]{article}
        \usepackage[latin1]{inputenc}
        \usepackage[portuguese]{babel}
        \usepackage[T1]{fontenc}
        \usepackage{amsmath}
        \usepackage{amsfonts}
        \usepackage{amssymb}
        \usepackage{graphicx}
        \usepackage[left=2.5cm,right=2.5cm,top=2.5cm,bottom=2.5cm]{geometry}
        \author{Felipe Veiga Ramos}
        \title{}

        \begin{document}

                \begin{titlepage}
                \LARGE
                        \begin{center}
                        \vspace{5cm}
                        \textbf{Universidade Tecnol�gica Federal do Paran� \\ \vspace{1.8cm}}
                        \includegraphics[scale=0.35]{logoutfpr.jpg} \\ \vspace{1.8cm}
                        \textit{Engenharia de Software II} \vspace{2cm} \\
                        Felipe veiga Ramos \\ 1061461 \vspace{2cm} \\
                        PROJETO 4 - ESTIMATIVA DE TEMPO PARA DESENVOLVIMENTO\vspace{2cm} \\
                        Departamento de Ci�ncia da Computa��o (DACOM)

                        \end{center}
                \end{titlepage}

                        \tableofcontents
                        \newpage
                        \section{Da Natureza Deste Documento}
                        \paragraph{}
Este documento objetiva estimar o tempo e esfor�o
necess�rio para a implementa��o do CV-Lates, como descrito em documentos
apropriados.



        \paragraph{}
Para tal, foi utilizado a t�cnica \textbf{EAP} (Estrutura Anal�tica de
Projeto, do ingl�s \textit{Work breakdown structure}) e na f�rmula \textbf{PERT}
(\textit{Program Evaluation and Review Technique}). A seguir ser�o apresentadas as subdivis�es do trabalho em etapas. Estas s�o baseadas na leitura feita a partir
dos requisitos funcionais levantados atrav�s da documenta��o do projeto.



        \section{Escopo do Projeto}
        \begin{itemize}
                \item Etapa 1: Diagrama��o de casos de uso
                \begin{itemize}
                        \item Otimista: 2 horas
                        \item Pessimista: 5 horas
                        \item Mais prov�vel: 3 horas
                        \item Tempo esperado: 3,16 horas
                \end{itemize}
                \item Etapa 2: Diagrama��o de classes

                \begin{itemize}
                        \item Otimista: 1 hora
                        \item Pessimista: 5 horas
                        \item Mais prov�vel:2 horas
                        \item Tempo esperado: 2,33 horas
                \end{itemize}
                        \item Etapa 3: Implementa��o de classes
                \begin{itemize}
                                \item Otimista: 10 horas
                                \item Pessimista: 30 horas
                                \item Mais prov�vel: 15 horas
                                \item Tempo Esperado: 16.6 horas

                \end{itemize}
         \item Etapa 4: Implementa��o de opera��es de Banco de Dados
                        \begin{itemize}
                                \item Otimista: 6 horas
                                \item Pessimista: 10 horas
                                \item Mais prov�vel: 7 horas
                                \item Tempo esperado: 7,33 horas
                        \end{itemize}
                        \item Etapa 5: Implementa��o de Interface gr�fica
                        \begin{itemize}
                                \item Otimista: 20 horas
                                \item Pessimista: 40 horas
                                \item Mais prov�vel: 30 horas
                                \item Tempo esperado: 30 horas
                        \end{itemize}
                        \item Etapa 6: Realiza��o de testes
                        \begin{itemize}
                                \item Otimista: 3 horas
                                \item Pessimista: 14 horas
                                \item Mais prov�vel: 5 horas
                                \item Tempo esperado: 4 horas
                        \end{itemize}
                        \item Etapa 7: Corre��es
                        \begin{itemize}
                                \item Otimista: 4 horas
                                \item Pessimista: 15 horas
                                \item Mais esperado: 8 horas
                                \item Tempo esperado: 8,5 horas
                        \end{itemize}
        \end{itemize}

        \section{Tempo Total Esperado: Segundo a previs�o}

        \paragraph{}
Apresentada a metodologia detalhada bem como a estimativa
por atividade, a estimativa total ser� de 71,92 (aprox. 72 horas).



        \section{Justificativa: justificando a metologia empregada para previs�es}

        \paragraph{}

Esta metodologia foi empregada na realiza��o da estimativa de tempo de trabalho, pois ela depende  do conhecimento - em sua maioria conceitual - do
estimador sobre o projeto a ser trabalhado e sobre os membros de sua equipe. Al�m disso, o tamanho do projeto (complexidade) tamb�m � considerada.


                \end{document}
